\documentclass[12pt,a4paper]{article}
\usepackage[utf8]{inputenc}
\usepackage[brazilian]{babel}
\usepackage{graphicx}
\usepackage{listings}
\usepackage{xcolor}
\usepackage{hyperref}
\usepackage{amsmath}
\usepackage{geometry}
\usepackage{fancyhdr}
\usepackage{titlesec}
\usepackage{enumitem}
\usepackage{tcolorbox}
\usepackage{fontawesome5}
\usepackage{tikz}

\geometry{left=2.5cm,right=2.5cm,top=2.5cm,bottom=2.5cm}

% Definir cores personalizadas
\definecolor{primaryblue}{RGB}{41,128,185}
\definecolor{darkblue}{RGB}{23,32,42}
\definecolor{lightblue}{RGB}{174,214,241}
\definecolor{successgreen}{RGB}{39,174,96}
\definecolor{warningorange}{RGB}{230,126,34}
\definecolor{dangerred}{RGB}{231,76,60}
\definecolor{codebg}{RGB}{245,245,245}
\definecolor{commentgreen}{RGB}{46,125,50}
\definecolor{keywordpurple}{RGB}{123,31,162}
\definecolor{stringorange}{RGB}{255,87,34}

% Configuração avançada do código Python
\lstdefinestyle{pythonstyle}{
    language=Python,
    backgroundcolor=\color{codebg},
    commentstyle=\color{commentgreen}\itshape,
    keywordstyle=\color{keywordpurple}\bfseries,
    numberstyle=\tiny\color{gray},
    stringstyle=\color{stringorange},
    basicstyle=\ttfamily\small,
    breakatwhitespace=false,
    breaklines=true,
    captionpos=b,
    keepspaces=true,
    numbers=left,
    numbersep=8pt,
    showspaces=false,
    showstringspaces=false,
    showtabs=false,
    tabsize=4,
    frame=leftline,
    framesep=10pt,
    framerule=2pt,
    rulecolor=\color{primaryblue},
    xleftmargin=15pt,
    framexleftmargin=10pt
}

\lstdefinestyle{bashstyle}{
    language=bash,
    backgroundcolor=\color{darkblue!5},
    commentstyle=\color{commentgreen}\itshape,
    keywordstyle=\color{primaryblue}\bfseries,
    numberstyle=\tiny\color{gray},
    stringstyle=\color{successgreen},
    basicstyle=\ttfamily\small\color{darkblue},
    breakatwhitespace=false,
    breaklines=true,
    keepspaces=true,
    showspaces=false,
    showstringspaces=false,
    showtabs=false,
    tabsize=4,
    frame=single,
    framesep=8pt,
    rulecolor=\color{darkblue!30}
}

\lstset{style=pythonstyle}

% Caixas personalizadas
\newtcolorbox{infobox}[1]{
    colback=lightblue!10,
    colframe=primaryblue,
    fonttitle=\bfseries,
    title={\faInfoCircle\ #1},
    arc=3mm
}

\newtcolorbox{warningbox}[1]{
    colback=warningorange!10,
    colframe=warningorange,
    fonttitle=\bfseries,
    title={\faExclamationTriangle\ #1},
    arc=3mm
}

\newtcolorbox{successbox}[1]{
    colback=successgreen!10,
    colframe=successgreen,
    fonttitle=\bfseries,
    title={\faCheckCircle\ #1},
    arc=3mm
}

\newtcolorbox{dangerbox}[1]{
    colback=dangerred!10,
    colframe=dangerred,
    fonttitle=\bfseries,
    title={\faTimesCircle\ #1},
    arc=3mm
}

% Configuração de seções
\titleformat{\section}
    {\normalfont\LARGE\bfseries\color{primaryblue}}
    {\thesection}{1em}{}
    [\titlerule]

\titleformat{\subsection}
    {\normalfont\Large\bfseries\color{darkblue}}
    {\thesubsection}{1em}{}

\titleformat{\subsubsection}
    {\normalfont\large\bfseries\color{primaryblue!80}}
    {\thesubsubsection}{1em}{}

% Cabeçalho e rodapé personalizados
\pagestyle{fancy}
\fancyhf{}
\fancyhead[L]{\color{primaryblue}\textbf{Segurança Computacional}}
\fancyhead[R]{\color{darkblue}AES \& RSA}
\fancyfoot[C]{\color{gray}\thepage}
\renewcommand{\headrulewidth}{2pt}
\renewcommand{\headrule}{\hbox to\headwidth{\color{primaryblue}\leaders\hrule height \headrulewidth\hfill}}

% Configuração de hyperlinks
\hypersetup{
    colorlinks=true,
    linkcolor=primaryblue,
    filecolor=primaryblue,
    urlcolor=primaryblue,
    citecolor=primaryblue
}

\title{
    \Huge\textbf{\color{primaryblue}Sistema de Criptografia}\\
    \vspace{5mm}
    \Large\textbf{AES e RSA}\\
    \vspace{3mm}
    \large Implementação e Documentação\\
    \vspace{2mm}
    \normalsize Trabalho 01 - Segurança em Sistemas
}
\author{
    \large Caio Victor Ferreira do Nascimento\\
    \normalsize caio.ferreira@ufpi.edu.br\\
    \normalsize Universidade Federal do Piauí - UFPI\\
    \normalsize Disciplina: Segurança em Sistemas
}
\date{\today}

\begin{document}

\maketitle
\thispagestyle{empty}

\vspace{15mm}
\begin{center}
\begin{tcolorbox}[width=0.85\textwidth,colback=primaryblue!5,colframe=primaryblue,arc=5mm]
\centering
\Large\textbf{Resumo}\\
\vspace{3mm}
\normalsize
Este documento apresenta uma implementação completa de algoritmos criptográficos em Python, incluindo \textbf{AES} (cifragem simétrica nos modos ECB e CBC) e \textbf{RSA} (geração de chaves, assinatura e verificação digital). O sistema oferece interface interativa via terminal com suporte a múltiplos formatos de entrada/saída (HEX, Base64, UTF-8) e compatibilidade total com OpenSSL.
\end{tcolorbox}
\end{center}

\newpage

\section{Introdução}

Este sistema implementa os algoritmos \textbf{AES} e \textbf{RSA} em Python usando a biblioteca \texttt{cryptography}, oferecendo funcionalidades de cifragem simétrica, geração de chaves assimétricas, assinatura digital e verificação.

\subsection{Objetivos}

\begin{itemize}[leftmargin=*]
    \item[\faCheck] Implementar AES nos modos ECB e CBC com padding PKCS7
    \item[\faCheck] Gerar pares de chaves RSA (1024 e 2048 bits)
    \item[\faCheck] Implementar assinatura digital RSA-PSS com SHA-2
    \item[\faCheck] Implementar assinatura digital RSA (PKCS\#1 v1.5) com SHA-2
    \item[\faCheck] Fornecer interface interativa e intuitiva
\end{itemize}

\subsection{Requisitos}

\begin{tcolorbox}[colback=darkblue!5,colframe=darkblue!50]
\textbf{Software:} Python 3.6+ e biblioteca \texttt{cryptography}\\
\textbf{Instalação:} \texttt{pip install cryptography}
\end{tcolorbox}

\section{Fundamentos}

O sistema implementa \textbf{AES} (cifragem simétrica nos modos ECB e CBC com chaves de 128/192/256 bits) e \textbf{RSA} (geração de chaves 1024/2048 bits, assinatura digital com PKCS\#1 v1.5 e SHA-2). Modo CBC requer IV de 128 bits e é mais seguro que ECB. Formato de chaves RSA é PEM, compatível com OpenSSL.

\section{Execução do Sistema}

\subsection{Como Executar}

\begin{lstlisting}[style=bashstyle]
python trabalhoSeg.py
\end{lstlisting}

\begin{tcolorbox}[colback=darkblue!90,coltext=white]
\footnotesize
\begin{verbatim}
============================================================
  SISTEMA DE CRIPTOGRAFIA - AES E RSA
============================================================
1. Cifrar arquivo (AES)
2. Decifrar arquivo (AES)
3. Gerar par de chaves RSA
4. Assinar arquivo (RSA)
5. Verificar assinatura (RSA)
0. Sair
============================================================
Escolha uma opção: _
\end{verbatim}
\end{tcolorbox}

\subsection{Opção 1: Cifrar Arquivo (AES)}

O sistema solicitará 8 informações em sequência:

\begin{tcolorbox}[colback=primaryblue!10,colframe=primaryblue,title={\faLock\ Campos para Cifragem AES}]

\textbf{1. Arquivo de entrada (claro):}\\
\quad \faFileAlt\ Digite o nome do arquivo que deseja cifrar\\
\quad Exemplo: \texttt{mensagem.txt} ou \texttt{foto.jpg}

\textbf{2. Chave:}\\
\quad \faKey\ Digite a chave secreta (string de texto ou hexadecimal)\\
\quad \textit{HEX}: \texttt{3f2a7b9c4d8e1f5a6b3c9d2e4f7a8b1c} (32 chars = 128 bits)\\
\quad \textit{UTF8}: \texttt{SenhaSegura123!} (texto comum)

\textbf{3. Tamanho da chave (128/192/256):}\\
\quad \faHashtag\ Digite \texttt{128}, \texttt{192} ou \texttt{256}\\
\quad $\rightarrow$ A chave fornecida deve ter exatamente:\\
\quad \quad 128 bits = 16 bytes = 32 chars hex\\
\quad \quad 192 bits = 24 bytes = 48 chars hex\\
\quad \quad 256 bits = 32 bytes = 64 chars hex

\textbf{4. Modo de operação (ECB/CBC):}\\
\quad \faCogs\ Digite \texttt{ECB} ou \texttt{CBC}\\
\quad \textit{ECB}: Modo simples, não precisa de IV\\
\quad \textit{CBC}: Modo seguro, requer IV (próximo campo)

\textbf{5. Vetor de Inicialização (IV):} \textit{[apenas se CBC]}\\
\quad \faRandom\ Digite o IV de 16 bytes (128 bits)\\
\quad \textit{HEX}: \texttt{9a8b7c6d5e4f3a2b1c0d9e8f7a6b5c4d} (32 chars)\\
\quad \textit{UTF8}: \texttt{InicioAleatorio!} (16 caracteres)

\textbf{6. Formato da chave/IV (HEX/UTF8):}\\
\quad \faCode\ Digite \texttt{HEX} ou \texttt{UTF8}\\
\quad Define como interpretar a chave e IV fornecidos acima

\textbf{7. Formato de saída (HEX/BASE64):}\\
\quad \faFileCode\ Digite \texttt{HEX} ou \texttt{BASE64}\\
\quad Define como o texto cifrado será salvo no arquivo de saída

\textbf{8. Arquivo de saída:}\\
\quad \faSave\ Digite o nome do arquivo cifrado\\
\quad Exemplo: \texttt{mensagem\_cifrada.txt}

\end{tcolorbox}

\begin{warningbox}
\textbf{\faExclamationTriangle\ IMPORTANTE:} Guarde a chave, IV e modo utilizados! Sem eles, não será possível decifrar.
\end{warningbox}

\subsection{Opção 2: Decifrar Arquivo (AES)}

\textbf{Use EXATAMENTE os mesmos parâmetros da cifragem:}

\begin{tcolorbox}[colback=successgreen!10,colframe=successgreen,title={\faUnlock\ Campos para Decifragem AES}]

\textbf{1. Arquivo de entrada (cifrado):}\\
\quad \faFileAlt\ Digite o nome do arquivo cifrado\\
\quad Exemplo: \texttt{mensagem\_cifrada.txt}

\textbf{2. Chave:}\\
\quad \faKey\ Digite a \textbf{mesma chave} usada na cifragem

\textbf{3. Tamanho da chave (128/192/256):}\\
\quad \faHashtag\ Digite o \textbf{mesmo tamanho} usado na cifragem

\textbf{4. Modo de operação (ECB/CBC):}\\
\quad \faCogs\ Digite o \textbf{mesmo modo} usado na cifragem

\textbf{5. Vetor de Inicialização (IV):} \textit{[se usou CBC]}\\
\quad \faRandom\ Digite o \textbf{mesmo IV} usado na cifragem

\textbf{6. Formato da chave/IV (HEX/UTF8):}\\
\quad \faCode\ Digite o \textbf{mesmo formato} usado na cifragem

\textbf{7. Formato de entrada (HEX/BASE64):}\\
\quad \faFileCode\ Digite o formato do arquivo cifrado\\
\quad Deve corresponder ao "Formato de saída" da cifragem

\textbf{8. Arquivo de saída:}\\
\quad \faSave\ Digite o nome para salvar o arquivo decifrado\\
\quad Exemplo: \texttt{mensagem\_recuperada.txt}

\end{tcolorbox}

\subsection{Opção 3: Gerar Chaves RSA}

Cria um par de chaves (pública + privada) para assinatura digital:

\begin{tcolorbox}[colback=warningorange!10,colframe=warningorange,title={\faKey\ Campos para Geração RSA}]

\textbf{1. Tamanho da chave (1024/2048):}\\
\quad \faHashtag\ Digite \texttt{1024} ou \texttt{2048}\\
\quad Recomendado: \texttt{2048} (mais seguro)

\textbf{2. Arquivo para chave privada:}\\
\quad \faLock\ Digite o nome do arquivo (ex: \texttt{privada.pem})\\
\quad \textbf{Mantenha este arquivo em segredo!}

\textbf{3. Arquivo para chave pública:}\\
\quad \faUnlock\ Digite o nome do arquivo (ex: \texttt{publica.pem})\\
\quad Este arquivo pode ser compartilhado

\end{tcolorbox}

\subsection{Opção 4: Assinar Arquivo (RSA)}

Cria uma assinatura digital usando a \textbf{chave privada}:

\begin{tcolorbox}[colback=primaryblue!10,colframe=primaryblue,title={\faSignature\ Campos para Assinatura RSA}]

\textbf{1. Arquivo com chave privada:}\\
\quad \faLock\ Digite o nome do arquivo da chave privada\\
\quad Exemplo: \texttt{privada.pem}

\textbf{2. Arquivo em claro:}\\
\quad \faFileAlt\ Digite o nome do arquivo que deseja assinar\\
\quad Exemplo: \texttt{documento.txt}

\textbf{3. Arquivo para assinatura:}\\
\quad \faSave\ Digite o nome para salvar a assinatura\\
\quad Exemplo: \texttt{documento.sig}

\textbf{4. Versão SHA-2 (256/384/512):}\\
\quad \faHashtag\ Digite \texttt{256}, \texttt{384} ou \texttt{512}\\
\quad Recomendado: \texttt{256} (mais comum)

\textbf{5. Formato de saída (HEX/BASE64):}\\
\quad \faFileCode\ Digite \texttt{HEX} ou \texttt{BASE64}\\
\quad Define como a assinatura será salva

\end{tcolorbox}

\subsection{Opção 5: Verificar Assinatura (RSA)}

Valida se a assinatura é autêntica usando a \textbf{chave pública}:

\begin{tcolorbox}[colback=successgreen!10,colframe=successgreen,title={\faCheckCircle\ Campos para Verificação RSA}]

\textbf{1. Arquivo com chave pública:}\\
\quad \faUnlock\ Digite o nome do arquivo da chave pública\\
\quad Exemplo: \texttt{publica.pem}

\textbf{2. Arquivo em claro:}\\
\quad \faFileAlt\ Digite o nome do arquivo original (não cifrado)\\
\quad Deve ser o \textbf{mesmo arquivo} que foi assinado

\textbf{3. Arquivo com assinatura:}\\
\quad \faFileSignature\ Digite o nome do arquivo de assinatura\\
\quad Exemplo: \texttt{documento.sig}

\textbf{4. Versão SHA-2 (256/384/512):}\\
\quad \faHashtag\ Digite a \textbf{mesma versão} usada na assinatura

\textbf{5. Formato da assinatura (HEX/BASE64):}\\
\quad \faFileCode\ Digite o \textbf{mesmo formato} usado na assinatura

\end{tcolorbox}

\textbf{Resultado:} O sistema exibirá \texttt{✓ Assinatura VÁLIDA} ou \texttt{✗ Assinatura INVÁLIDA}

\begin{tcolorbox}[colback=lightblue!5,colframe=primaryblue,title={Uso no CyberChef}]
\small
Se você for verificar assinaturas usando o CyberChef, siga estas recomendações rápidas:

\begin{itemize}
    \item Se a assinatura foi salva em \textbf{BASE64}: cole o conteúdo do arquivo de assinatura no \textbf{Input} do CyberChef e adicione a operação \texttt{From Base64} antes de \texttt{RSA Verify}.
    \item Se a assinatura foi salva em \textbf{HEX}: cole o conteúdo do arquivo de assinatura no \textbf{Input} e adicione \texttt{From Hex} antes de \texttt{RSA Verify}.
    \item Configure a operação \texttt{RSA Verify} com a chave pública (PEM), o esquema de padding (no nosso caso: \textbf{PKCS#1 v1.5}) e o algoritmo de hash correspondente (SHA-256/384/512).
    \item Exemplo de receita no CyberChef (para assinatura em BASE64 com SHA-256):
\begin{verbatim}
From Base64 -> RSA Verify (Public Key: publica.pem, Padding: PKCS1v15, Hash: SHA-256)
\end{verbatim}
    \item Para HEX apenas troque \texttt{From Base64} por \texttt{From Hex} na receita acima.
    \item Alternativamente, se o CyberChef permitir especificar o formato de entrada na própria operação \texttt{RSA Verify}, selecione o formato correto (HEX/BASE64) em vez de usar \texttt{From Base64}/\texttt{From Hex}.
\end{itemize}

        extbf{Observação:} garanta que o arquivo de mensagem usado para verificar é exatamente o mesmo (mesmo conteúdo/encoding/EOL) que foi assinado.
\end{tcolorbox}

\subsection{Exemplo Completo: Workflow AES}

\begin{lstlisting}[style=bashstyle,caption={1. Preparar arquivo}]
echo "Dados confidenciais da empresa" > relatorio.txt
\end{lstlisting}

\textbf{2. No menu, escolha opção \texttt{1} e preencha:}

\begin{tcolorbox}[colback=codebg,colframe=gray!50]
\footnotesize
\texttt{Arquivo de entrada: relatorio.txt}\\
\texttt{Chave: ChaveSecreta2024!!}\\
\texttt{Tamanho da chave: 128} \textcolor{gray}{(16 caracteres)}\\
\texttt{Modo: CBC}\\
\texttt{IV: InicioAleatori!} \textcolor{gray}{(16 caracteres)}\\
\texttt{Formato chave/IV: UTF8}\\
\texttt{Formato de saída: BASE64}\\
\texttt{Arquivo de saída: relatorio\_cifrado.txt}
\end{tcolorbox}

\textbf{3. Para decifrar, escolha opção \texttt{2} com os mesmos parâmetros:}

\begin{tcolorbox}[colback=codebg,colframe=gray!50]
\footnotesize
\texttt{Arquivo de entrada: relatorio\_cifrado.txt}\\
\texttt{[mesmos parâmetros da cifragem]}\\
\texttt{Arquivo de saída: relatorio\_recuperado.txt}
\end{tcolorbox}

\subsection{Exemplo Completo: Workflow RSA}

\begin{lstlisting}[style=bashstyle,caption={1. Preparar documento}]
echo "Contrato de venda no valor de R\$ 100.000" > contrato.txt
\end{lstlisting}

\textbf{2. No menu, escolha opção \texttt{3} para gerar chaves:}

\begin{tcolorbox}[colback=codebg,colframe=gray!50]
\footnotesize
\texttt{Tamanho da chave: 2048}\\
\texttt{Arquivo para chave privada: privada.pem}\\
\texttt{Arquivo para chave pública: publica.pem}
\end{tcolorbox}

\textbf{3. No menu, escolha opção \texttt{4} para assinar o documento:}

\begin{tcolorbox}[colback=codebg,colframe=gray!50]
\footnotesize
\texttt{Arquivo com chave privada: privada.pem}\\
\texttt{Arquivo em claro: contrato.txt}\\
\texttt{Arquivo para assinatura: contrato.sig}\\
\texttt{Versão SHA-2: 256}\\
\texttt{Formato de saída: BASE64}
\end{tcolorbox}

\textbf{4. Para verificar, escolha opção \texttt{5}:}

\begin{tcolorbox}[colback=codebg,colframe=gray!50]
\footnotesize
\texttt{Arquivo com chave pública: publica.pem}\\
\texttt{Arquivo em claro: contrato.txt}\\
\texttt{Arquivo com assinatura: contrato.sig}\\
\texttt{Versão SHA-2: 256}\\
\texttt{Formato da assinatura: BASE64}\\[0.5em]
\textcolor{successgreen}{\textbf{Resultado: ✓ Assinatura VÁLIDA}}
\end{tcolorbox}

\textbf{5. Teste de integridade - modifique o arquivo:}

\begin{lstlisting}[style=bashstyle]
echo "FRAUDE" >> contrato.txt
\end{lstlisting}

\textbf{6. Verifique novamente (opção \texttt{5} com mesmos parâmetros):}

\begin{tcolorbox}[colback=dangerred!10,colframe=dangerred]
\footnotesize
\textcolor{dangerred}{\textbf{Resultado: ✗ Assinatura INVÁLIDA}}\\
\textit{O sistema detectou que o arquivo foi alterado!}
\end{tcolorbox}

\subsection{Dicas Práticas de Uso}

\begin{infobox}
\textbf{Formatos de Chave/IV:}
\begin{itemize}
    \item \textbf{HEX:} Use quando gerar chaves aleatórias com Python\\
    \quad Exemplo: \texttt{3f2a7b9c4d8e1f5a...} (caracteres 0-9, a-f)
    \item \textbf{UTF8:} Use quando quiser senhas legíveis\\
    \quad Exemplo: \texttt{MinhaSenha123!} (qualquer texto)
\end{itemize}
\end{infobox}

\begin{infobox}
\textbf{Formatos de Saída:}
\begin{itemize}
    \item \textbf{HEX:} Compatível com ferramentas Unix (xxd, hexdump)
    \item \textbf{BASE64:} Compatível com email, JSON, OpenSSL
\end{itemize}
\end{infobox}

\begin{successbox}
\textbf{Tamanho de Chaves/IV:}\\[0.5em]
\begin{tabular}{lll}
\textbf{AES-128:} & 16 bytes & = 32 hex \textit{ou} 16 UTF-8 \\
\textbf{AES-192:} & 24 bytes & = 48 hex \textit{ou} 24 UTF-8 \\
\textbf{AES-256:} & 32 bytes & = 64 hex \textit{ou} 32 UTF-8 \\
\textbf{IV (CBC):} & 16 bytes & = 32 hex \textit{ou} 16 UTF-8 \\
\end{tabular}
\end{successbox}

\begin{warningbox}
\textbf{\faExclamationTriangle\ Lembre-se:}
\begin{itemize}
    \item \textbf{Cifragem AES:} Precisa da mesma chave/IV/modo para decifrar
    \item \textbf{Assinatura RSA:} Chave privada assina, chave pública verifica
    \item \textbf{Modo ECB:} Não precisa de IV, mas é menos seguro
    \item \textbf{Modo CBC:} Precisa de IV diferente para cada cifragem
\end{itemize}
\end{warningbox}

\section{Testes e Validação}

\subsection{Teste Rápido AES}

\begin{lstlisting}[style=bashstyle]
# 1. Criar arquivo de teste
echo "Mensagem de teste AES" > teste_aes.txt

# 2. Cifrar usando o sistema (opcao 1)
#    Chave: TesteCifragem!! (UTF8, 128 bits = 16 caracteres)
#    IV: InicializacaoX (UTF8, 16 caracteres)
#    Modo: CBC, Formato saida: BASE64

# 3. Decifrar usando mesma chave/IV (opcao 2)

# 4. Verificar se conteudo foi recuperado
diff teste_aes.txt teste_aes_decifrado.txt  # Deve estar vazio
\end{lstlisting}

\subsection{Teste Rápido RSA}

\begin{lstlisting}[style=bashstyle]
# 1. Criar documento de teste
echo "Documento de teste RSA" > teste_rsa.txt

# 2. Gerar par de chaves (opcao 3)
#    Tamanho: 2048 bits

# 3. Assinar documento (opcao 4)
#    SHA: 256, Formato: BASE64

# 4. Verificar assinatura (opcao 5)
#    Resultado esperado: "Assinatura VALIDA"
\end{lstlisting}

\section{Referências}

\begin{enumerate}
    \item STALLINGS, William. \textit{Cryptography and Network Security}. 7th ed. Pearson, 2017.
    \item NIST. \textit{Advanced Encryption Standard (AES)}. FIPS PUB 197, 2001.
    \item Python Cryptographic Authority. \textit{Cryptography Documentation}. \url{https://cryptography.io/}, 2025.
\end{enumerate}

\vspace{10mm}

\begin{center}
\begin{tcolorbox}[width=0.85\textwidth,colback=primaryblue!10,colframe=primaryblue]
\textbf{Contato}\\
Caio Victor Ferreira do Nascimento\\
caio.ferreira@ufpi.edu.br\\
Universidade Federal do Piauí - UFPI\\
Disciplina: Segurança em Sistemas\\
Outubro de 2025
\end{tcolorbox}
\end{center}

\end{document}
